\documentclass{article}
 \usepackage[utf8]{inputenc}
 \usepackage[T1]{fontenc}
 \usepackage{graphicx}
 
 % Used in the explanation text
 \usepackage[colorlinks]{hyperref}
 
 % Used by the template
 \usepackage{setspace}
 \usepackage{pgf}
 \usepackage{changepage} % to adjust margins
 \usepackage[breakable]{tcolorbox}
 \usepackage{float}
 \usepackage{enumitem}
 
 \title{Model cards - Tweeter Gender Classification}
 \author{Vijaya lakshmi Kuruba - Houda Saadaoui}
 
 \date{March 2021}
 
 \begin{document}
 
 \maketitle
 
 
 % Each section is supposed to be brief, in the form of a bullet list.
 % This environment formats the lists in each model card section in a compact format to help
 % the card fit into the recommended "one to two pages".
 \newenvironment{mcsection}[1]
     {%
         \textbf{#1}
 
         % Reduce margins to use the space more effectively and help fit in the recommended "one to two pages"
         % Use the bullet list format as shown in the model card paper to increase readability
         \begin{itemize}[leftmargin=*,topsep=0pt,itemsep=-1ex,partopsep=1ex,parsep=1ex,after=\vspace{\medskipamount}]
     }
     {%
         \end{itemize}
     }
 
 % Optional: reduce margins single line to fit in "one to two pages", as recommended
 \begin{adjustwidth}{-60pt}{-30pt}
 \begin{singlespace}
 
 \tcbset{colback=white!10!white}
 \begin{tcolorbox}[title=\textbf{Model Card - Tweeter Gender Classification },
     breakable, sharp corners, boxrule=0.7pt]
 
 % Change to a smaller, but still legible font size to help fit in the recommended "one to two pages"
 \small{
 
 \begin{mcsection}{Model Details}
    \item Developed by Houda Saadaoui and  Vijaya Lakshmi Kuruba as an academic requirement of the course IFT 6390 at the University of Montreal in Winter 2021
     \item  This is an exploratory SVM linear classifier which predicts the gender of the user based on its Tweets and profile description.
     \item SKlearn API is utilised for model creation.
     \item Released under MIT License
     \item For any questions about the model, contact houda.saadaoui@umontreal.ca,\\vijaya.lakshmi.kuruba@unomtreal.ca 

 \end{mcsection}
 
 \begin{mcsection}{Intended Use}
     \item Intended to be used for anyone interested  in the relationship between type of writing and gender. 
     \item Particularly intended for people who are interested in analysing social media use.
  

 \end{mcsection}
 
 \begin{mcsection}{Factors}
     \item The prediction of the model is within the scope of the social media users and particularly, Twitter user. 
     Therefore, relevant factors are age and level of education as most of the Twitter users are young and have higher education.
    \item Language is another relevant factor as the model was trained with English Tweets and English profile description

 \end{mcsection}
 
 \begin{mcsection}{Metrics}
     \item Model is evaluated with reporting the accuracy score of the classifer.
     
     \item A ROC curve is constructed by plotting the true positive rate (TPR) against the false positive rate (FPR). ROC response of different datasets, created from K-fold cross-validation.Ref[1]
     \item Model hyperparameter is tuned using RandomizedSearchCV by choosing the best alpha parameter in the followong list [1e-4, 1e-3, 1e-2, 1e-1, 1e0, 1e1, 1e2, 1e3]
     
 \end{mcsection}
 
 \begin{mcsection}{Training Data}
     \item This model uses the dataset for the CrowdFlower AI team.  The dataset contains 20,000 random tweets and account profile with the label corresponding to the gender of the user.
     \item Dataset was first filtred using only the gender ('male', 'female') as it has other irrelevent gender values
     \item Dataset was preprocessed using the tweet-preprocessor library 
     \item Dataset is split to 80-20 ratio to account for train and test data
     \item Text features where vectorized using all the dataset vocabulary 

 \end{mcsection}
     \includegraphics[scale=0.45]{Crowd.png}
    \includegraphics[scale=0.4]{feature_Importance.png}
 
 \begin{mcsection}{Evaluation Data}
   
     \item Data is split to 80-20 ratio to account for train and test data
     \item The model has an accuracy of ACCDONOTMODIFY \%
  

 \end{mcsection}
 
 
 \begin{mcsection}{Ethical Considerations}
     \item Because of the population in which the study was conducted, the results should be carefully generalized to more than one population.
     \item In no case can this model be used to make decisions involving gender discrimination.
 \end{mcsection}
 
 \begin{mcsection}{Caveats and Recommendations}
     \item We have considered only tweets and profile description for classification
     \item Model performence can be improved by including more features based on exploratory analysis.
      \item Model performence can be improved by using a larger dataset.
 \end{mcsection}
 
 \textbf{Quantitative Analyses}
 
 \includegraphics[scale=0.28]{Loss.png} \includegraphics[scale=0.28]{Confusion-matrix.png}
 \includegraphics[scale=0.28]{ROC_plot_Kfold.png}
 % Sample table inside tcolorbox

 
 } % end font size change
 \end{tcolorbox}
 \end{singlespace}
 \end{adjustwidth}
 
 
 
 \end{document}
 
